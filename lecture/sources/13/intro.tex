\begin{frame}{Learning without supervision?}

  \begin{itemize}
    \item 1,281,167 training images
    \item 1000 object classes
  \end{itemize}

  \note{
    \begin{itemize}
      \item How much are 1000 concepts compared to all the concepts humans use?
      \item Imagine we would need to label 1000 images per concept.
      \item New concepts are created and change all the time.
    \end{itemize}
  }

\end{frame}


\begin{frame}{Learning without supervision?}

  \begin{itemize}
    \item Can we learn without a supervision signal in form of labels?
    \item In an un- or rather self-supervised manner?
  \end{itemize}

  \note{
    \begin{itemize}
      \item Similar to a human child in the first few month after birth.
      \item Purely by observing the world.
      \item It's hard to define what truly unsupervised learning could be. Therefore the term self-supervised learning is a better fit.
    \end{itemize}
  }

\end{frame}


\begin{frame}{Learning without supervision?}

  \begin{itemize}
    \item Pretext tasks
    \item Energy based methods
    \item Generative learning
  \end{itemize}

  \note{
    \begin{itemize}
      \item We will look at three big topics today.
      \item At least the second and third topic could not only fill a lecture but a full course on their own.
      \item E.g. CS 236: Deep Generative Models (Stanford) or CS 294-158 Deep Unsupervised Learning (Berkeley)
    \end{itemize}
  }

\end{frame}


\begin{frame}{Learning with self-supervision}

  Idea:
  \begin{itemize}
    \item Train a neural network with an objective that doesn't need labels.
    \item Evaluate representation on a downstream task. E.g. performance on ImageNet with or without finetuning.
  \end{itemize}

  \note{
    \begin{itemize}
      \item In generative learning often, people often just want to generate visual content though.
    \end{itemize}
  }

\end{frame}
